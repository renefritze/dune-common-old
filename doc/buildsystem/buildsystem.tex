\documentclass[11pt,a4paper,headinclude,footinclude,DIV16,normalheadings]{scrartcl}
\usepackage[automark]{scrpage2}
\usepackage[ansinew]{inputenc}
%\usepackage{german}
%\usepackage{bibgerm}
\usepackage{amsmath}
\usepackage{amsfonts}
\usepackage{theorem}
\usepackage{color}
\usepackage{listings}
\lstset{language=C++, basicstyle=\ttfamily,
  keywordstyle=\color{black}\bfseries, tabsize=4,
  stringstyle=\ttfamily, commentstyle=\it, extendedchars=true}
\usepackage{hyperref}
\usepackage{psfrag}
\usepackage{pstricks}
\usepackage{makeidx}
\usepackage{xspace}

\usepackage{graphicx}

\newtheorem{lst}{Listing}

\newcommand{\dune}{\texttt{DUNE}\xspace}
\newcommand{\autoconf}{\texttt{autoconf}\xspace}
\newcommand{\automake}{\texttt{automake}\xspace}
\newcommand{\autogen}{\texttt{autogen.sh}\xspace}
\newcommand{\libtool}{\texttt{libtool}\xspace}
\newcommand{\configure}{\texttt{configure}\xspace}
\newcommand{\configureac}{\texttt{configure.ac}\xspace}
\newcommand{\makefile}{\texttt{Makefile}\xspace}
\newcommand{\makefilein}{\texttt{Makefile.in}\xspace}
\newcommand{\makefileam}{\texttt{Makefile.am}\xspace}
\newcommand{\dunecommon}{\texttt{dune-common}\xspace}
\newcommand{\duneistl}{\texttt{dune-istl}\xspace}
\newcommand{\dunegrid}{\texttt{dune-grid}\xspace}
\newcommand{\dunedisc}{\texttt{dune-disc}\xspace}
\newcommand{\dunecontrol}{\texttt{dunecontrol}\xspace}
\newcommand{\dunemodule}{\texttt{dune.module}\xspace}
\newcommand{\make}{\texttt{make}\xspace}

\pagestyle{scrheadings}

\title{The DUNE Buildsystem HOWTO}

\author{Christian Engwer$^\ast$}

\date{\today}

\publishers{%
{\normalsize $^\ast$Interdisziplin�res Zentrum f�r Wissenschaftliches Rechnen,
Universit�t Heidelberg,\\
Im Neuenheimer Feld 368, D-69120 Heidelberg, Germany}\\
\bigskip
{\normalsize \texttt{\url{http://hal.iwr.uni-heidelberg.de/dune/index.html}}}\\
}

\begin{document}

\maketitle
\tableofcontents
\pagebreak
\section{Structure of DUNE}
\dune consists of several independent modules:
\begin{itemize}
\item \dunecommon
\item \dunegrid
\item \duneistl
\item \dunedisc
\item \texttt{dune-fem}
\end{itemize}

These modules interact very tightly and depend on each
other.

The build system is structured as follows:
\begin{itemize}
\item Modules are build using the GNU autotools.
\item Each module has a set of modules it depends on, these modules
  have to be built before building the module itself.
\item Each module has a file \dunemodule which hold dependecies and
  other information regarding the module.
\item The modules can be built in the appropriate order using the
  \dunecontrol script (shippedwith \dunecommon)
\end{itemize}

\section{Toolchain}
\subsection{Autotools}

Software is generally developed to be used on multiple
platforms. Since each of these platforms have different compilers,
different include files, there is a need to write Makefiles and build
scripts so that they can work on a variety of platforms. The free
software community (Project GNU), faced with this problem, devised a
set of tools to generate Makefiles and build scripts that work on a
variety of platforms. If you have downloaded and built any GNU
software from source, you are familiar with the \configure script. The
\configure script runs a series of tests to determine information about
your machine.

The autotools simplify the generation of portable Makefiles and
configure scripts.

\minisec{autoconf}

\autoconf is used to create the \configure script. \configure is
created from \configureac, using a set of \texttt{m4} files.

\begin{center}
\psset{unit=0.5mm}
\psset{linewidth=0.4pt}
\begin{pspicture}(140,50)
\put(-26,45){\parbox{40\unitlength}{\centering{}\configureac}}
\put(116,45){\parbox{40\unitlength}{\centering{}\tt{}m4/*.m4}}
\put(0,36)
{
  \psset{linewidth=1pt}
  \psline[linearc=4]{c-}(-2,2)(-2,-2)(67.5,-2)(67.5,-6)
  \psline[linearc=4]{c-}(137,2)(137,-2)(67.5,-2)(67.5,-6)
  \psline{->}(67.5,-5.5)(67.5,-20)
}
\put(72,22){\textit{\autoconf}}
\put(50,7){\parbox{40\unitlength}{\centering{}\configure}}
\end{pspicture}
\end{center}

How to write a \configureac for \dune is described in \ref{configure.ac}

\minisec{automake}

\automake is used to create the \makefilein files (needed for
\configure) from \makefileam files and, using a set of include files
located in the \texttt{am} directory. These include files provide
additional features, not provided by the standard \automake (see
\ref{am_includes}). The \texttt{am} directory is in the \dunecommon
module and each module intending to use one of these includes has to
create symlink; this is usually done by \autogen (see
\ref{autogen.sh}).

\begin{center}
\psset{unit=0.5mm}
\psset{linewidth=0.4pt}
\begin{pspicture}(140,80)
\put(-26,75){\parbox{40\unitlength}{\centering{}\makefileam}}
\put(116,75){\parbox{40\unitlength}{\centering{}\tt{}am/*}}
\put(0,66)
{
  \psset{linewidth=1pt}
  \psline[linearc=4]{c-}(-2,2)(-2,-2)(67.5,-2)(67.5,-6)
  \psline[linearc=4]{c-}(137,2)(137,-2)(67.5,-2)(67.5,-6)
  \psline{->}(67.5,-5.5)(67.5,-20)
}
\put(72,52){\textit{\automake}}
\put(50,37){\parbox{40\unitlength}{\centering{}\texttt{Makefile.in}}}
\put(0,34)
{
  \psset{linewidth=1pt}
  \psline[linestyle=dashed,dash=1.5pt 1.5pt]{->}(67.5,0)(67.5,-14.5)
}
\put(72,25){\textit{\configure}}
\put(50,10){\parbox{40\unitlength}{\centering{}\texttt{Makefile}}}
\end{pspicture}
\end{center}

How to write a \configureac for \dune is described in \ref{configure.ac}

\minisec{libtool}

\libtool is used to build libraries on all platforms. This way we
don't have to deal will platform specific issues.

\subsection{Makefile.am}
\label{makefile.am}

\subsubsection{Overview}

Let's start off with a simple program {\em hello} built from
\texttt{hello.c}. As automake is designed to build and install a
package it needs to know

\begin{itemize}
\item what programs it should build
\item where to put them when installing
\item which sources to use
\end{itemize}

The \makefileam thus looks like this:

\begin{lstlisting}[basicstyle=\ttfamily\scriptsize]
noinst_PROGRAMS = hello
hello_SOURCES = hello.c
\end{lstlisting}

This would build hello and won't install it when \texttt{make
  install} is called. Using \lstinline!bin_PROGRAMS! instead of
\lstinline!noinst_PROGRAMS! would install the hello-binary into a
\texttt{textit{prefix}/bin} directory which we don't want to do with
most of the \dune applications.

Building more programs with a couple of source-files works like this

\begin{lstlisting}[basicstyle=\ttfamily\scriptsize]
noinst_PROGRAMS = hello bye

hello_SOURCES = common.c common.h hello.c
bye_SOURCES = common.c common.h bye.c parser.y lexer.l
\end{lstlisting}

automake has more integrated rules than the standard make, the example
above would automatically use yacc/lex to create
\texttt{parser.c/lexer.c} and build them into the {\em bye} binary.

Make-Variables may be defined and used as usual:

\begin{lstlisting}[basicstyle=\ttfamily\scriptsize]
noinst_PROGRAMS = hello bye

COMMON = common.c common.h

hello_SOURCES =  hello.c
bye_SOURCES =  bye.c parser.y lexer.l
\end{lstlisting}

Even normal make-rules may be used in a \makefileam.

\minisec{Using flags}

Compiler/linker/preprocessor-flags can be set either globally:

\begin{lstlisting}[basicstyle=\ttfamily\scriptsize]
noinst_PROGRAMS = hello bye

AM_CPPFLAGS = -DDEBUG

hello_SOURCES = hello.c
bye_SOURCES = bye.c
\end{lstlisting}

or locally:

\begin{lstlisting}[basicstyle=\ttfamily\scriptsize]
noinst_PROGRAMS = hello bye

hello_SOURCES = hello.c
hello_CPPFLAGS = -DHELLO

bye_SOURCES = bye.c
bye_CPPFLAGS = -DBYE
\end{lstlisting}

The local setting overrides the global one, thus

\begin{lstlisting}[basicstyle=\ttfamily\scriptsize]
hello_CPPFLAGS =  -Dmyflags
\end{lstlisting}

may be a good idea.

It is even possible to compile the same sources with different flags:

\begin{lstlisting}[basicstyle=\ttfamily\scriptsize]
noinst_PROGRAMS = hello bye

hello_SOURCES = generic-greeting.c
hello_CPPFLAGS = -DHELLO

bye_SOURCES = generic-greeting.c
bye_CPPFLAGS = -DBYE
\end{lstlisting}

Perhaps you're wondering why the above examples used
\texttt{AM\_CPPFLAGS} instead of the normal \texttt{CPPFLAGS}? The
reason for this is that the variables \texttt{CFLAGS},
\texttt{CPPFLAGS}, \texttt{CXXFLAGS} etc. are considered {\em user
  variables} which may be set on the commandline:

\begin{lstlisting}[basicstyle=\ttfamily\scriptsize]
make CXXFLAGS="-O2000"
\end{lstlisting}

This would override any settings in Makefile.am which might be
necessary to build. Thus, if the variables should be set even if the
user wishes to modify the values, you should use the \texttt{AM\_*}
version. 

The real compile-command always uses both \texttt{AM\_\textit{VAR}} and
\texttt{\it VAR}. Options that
autoconf finds are stored in the user variables (so that they may be
overridden) 

Commonly used variables are:
\begin{itemize}
\item \texttt{AM\_CPPFLAGS}: flags for the C-Preprocessor. This
  includes preprocessor defines like \texttt{-DNDEBUG} and include
  pathes like \texttt{-I/usr/local/package/include}
\item \texttt{AM\_CFLAGS}, \texttt{AM\_CXXFLAGS}: flags for the
  compiler (-g, -O, ...). One difference between these and the
  \texttt{CPPFLAGS} is that the linker will get
  \texttt{CFLAGS}/\texttt{CXXFLAGS} and \texttt{LDFLAGS} but not
  \texttt{CPPFLAGS}
\item \texttt{AM\_LDFLAGS} options for the linker
\item \texttt{LDADD}: libraries to link to a binary
\item \texttt{LIBADD}: libraries to add to a library
\item \texttt{SOURCES}: list of source-files (may include headers as well)
\end{itemize}

\minisec{Conditional builds}

Some parts of \dune only make sense if certain addon-packages were
found. autoconf therefore defines {\em conditionals} which automake can
use:

\begin{lstlisting}[basicstyle=\ttfamily\scriptsize]
if OPENGL
  PROGS = hello glhello
else
  PROGS = hello
endif

hello_SOURCES = hello.c

glhello_SOURCES = glhello.c hello.c
\end{lstlisting}

This will only build the {\em glhello} program if OpenGL was found. An
important feature of these conditionals is that they work with any
make program, even those without a native {\em if} construct like GNU-make.

\minisec{Default targets}

An automake-generated Makefile does not only know the usual {\em all},
{\em clean} and {\em install} targets but also
\begin{itemize}
\item {\bf tags} travel recursively through the directories and create
  TAGS-files which can be used in many editors to quickly find where
  symbols/functions are defined (use emacs-format)
\item {\bf ctags} the same as "tags" but uses the vi-format for the tags-files
\item {\bf dist} create a distribution tarball
\item {\bf distcheck} create a tarball and do a test-build if it really works
\end{itemize}

\minisec{Further documentation}

Further documentation no hwo to weite and maintain a \makefileam can
be found in the \automake manual
(\texttt{\url{http://www.gnu.org/software/automake/manual/}})

\subsubsection{Building Documentation}
\label{am_includes}

If you want to build documentation you might need additional make
rules. \dune offers a set of predefined rules to create certain kinds
of documentation. Therefor you have to include the appropriate rules
from the \texttt{am/} directory.

\minisec{html pages}
Webpages are created from wml sources, using the program \texttt{wml}
(\texttt{\url{http://thewml.org/}}).\\
\texttt{\$(top\_srcdir)/am/webstuff} containes the necessary rules.


\begin{lst}[File Makefile.am] \mbox{}

\lstinputlisting[basicstyle=\ttfamily\scriptsize,numbers=left,
numberstyle=\tiny, numbersep=5pt]{../Makefile.am}
\end{lst}

\minisec{\LaTeX documents}
In order to compile \LaTeX documents you can include
\texttt{\$(top\_srcdir)/am/latex}. This way you get rules for creation
of DVI files, PS files and PDF files.

\minisec{SVG graphics}
SVG graphics can be converted to png, in order to include them into
the web page. This conversion can be done using inkscape
(\texttt{\url{http://www.inkscape.org/}}).
\texttt{\$(top\_srcdir)/am/inkscape.am} offers the necessary rules.


\subsection{configure.ac}
\label{configure.ac}

\configureac  is a normal text file that contains several \autoconf
macros. These macros are evaluated my the \texttt{m4} macro processor
and transformed into a shell script.

\begin{lst}[File dune-common/configure.ac] \mbox{}

\lstinputlisting[basicstyle=\ttfamily\scriptsize,numbers=left,
numberstyle=\tiny, numbersep=5pt]{../../configure.ac}
\end{lst}

We offer a set of macros that can be used in your \configureac:

\begin{itemize}
\item \texttt{DUNE\_CHECK\_ALL\_M}
  runs all checks usually needed by a {\em \dune module}.
  This macros takes list of other \dune modules it should search for
  as parameters.
\begin{lstlisting}[basicstyle=\ttfamily\scriptsize]
DUNE_CHECK_ALL_M([dunecommon], [dunegrid])
\end{lstlisting}
  will search for \texttt{dune-common} and \texttt{dune-grid}
  (Attention: you have to provide the modules in such an order that
  the dependencies are checked already).
\item \texttt{DUNE\_CHECK\_ALL}
  same as \texttt{DUNE\_CHECK\_ALL\_M}, except that it only runs the
  tests needed for a {\em \dune application}
\item \texttt{DUNE\_SUMMARY\_ALL}
  prints information on the results of all major checks run by
  \texttt{DUNE\_CHECK\_ALL} or \texttt{DUNE\_CHECK\_ALL\_M}.
\end{itemize}

\texttt{DUNE\_CHECK\_ALL} and \texttt{DUNE\_CHECK\_ALL\_M} define certain
variables that can be used in the \configure script or in the
\makefileam:

\begin{itemize}
\item \texttt{DUNE\textit{\,MODULE\,}\_CPPFLAGS}
\item \texttt{DUNE\textit{\,MODULE\,}\_LDFLAGS}
\item \texttt{DUNE\textit{\,MODULE\,}\_LIBS}
\item \texttt{DUNE\textit{\,MODULE\,}ROOT}
\end{itemize}

For further information have a look at the autoconf manual \\
(\texttt{\url{http://www.gnu.org/software/autoconf/manual/}}).

\subsection{autogen.sh}
\label{autogen.sh}

\subsection{dunecontrol}
\label{dunecontrol}
\dunecontrol helps you building the different \dune modules in the
appropriate order. Each module has a \dunemodule file which contains
information on the module needed by \dunecontrol. 

\dunecontrol searches for \dunemodule files recursively from where you
are executing the program. For each \dune module found it will execute
a \dunecontrol command. All commands offered by \dunecontrol have a
default implementation. This default implementation can be overwriten
and extended in the \dunemodule file.

The commands you are interested in right now are
\begin{itemize}
\item \texttt{autogen} runs \autogensh for each module. A list of
  directories containing \dunemodule files and the parameters given on
  the commandline are passed as paramters to \autogensh.
\item \texttt{configure} runs \configure for each
  module. \texttt{--with-dune\textit{module}} parameters are created
  for a set of known \dune modules.
\item \texttt{make} runs \make for each module.
\item \texttt{all} runs \autogensh, \configure and \make for each module.
\end{itemize}

In order to build \dune the first time you will need the \texttt{all}
command. In pseudo code \texttt{all} does the following:
\begin{lstlisting}[basicstyle=\ttfamily\scriptsize]
foreach ($module in $Modules) {
  foreach (command in {autogen,configure,make) {
    run $command in $module
  }
}
\end{lstlisting}

This differs from calling
\begin{lstlisting}[basicstyle=\ttfamily\scriptsize]
dunecontrol autogen
dunecontrol configure
dunecontrol make
\end{lstlisting}
as it ensures that i.e. \dunecommon is fully built before \configure
is executed in \dunegrid. Otherwise \configure in \dunegrid would
complain that \texttt{libcommon.la} from \dunecommon is missing.

Further more you can add parameters to the commands; these parameters
get passed on to the program being executed. Assuming you want to call
\texttt{make clean} in all \dune modules you can execute
\begin{lstlisting}[basicstyle=\ttfamily\scriptsize]
dunecontrol make clean
\end{lstlisting}

\minisec{opts files}
You can also let \dunecontrol read the command parameters from a file.

\minisec{dune.module}

\begin{lst}[File dune.module] \mbox{}

\lstinputlisting[basicstyle=\ttfamily\scriptsize,numbers=left,
numberstyle=\tiny, numbersep=5pt]{../../dune.module}
\end{lst}



\section{Creating a new Dune module}

In order to create new \dune module, you have to provide 

\section{Creating a new Dune application}


\section{Futher questions}



\end{document}

%%% Local Variables: 
%%% mode: latex
%%% TeX-master: t
%%% End: 
